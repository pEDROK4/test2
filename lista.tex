\documentclass[12pt,reqno]{article}
\textwidth=14.5cm  \oddsidemargin=0.5cm
\usepackage{tikz}
\usepackage{graphicx}
\usepackage{psfrag}
\usepackage{mathrsfs}
\usepackage{color}
\usepackage{amsmath,amssymb}
\font\de=cmssi12
\font\dd=cmssi10
%%%%%%%%%%%%%%%%%%%%%%%%%%%%%%%%%%%%%%%%%%%%%%%%%%%%%%%%%%%%

\usepackage[active]{srcltx}

\numberwithin{equation}{section}

\newcommand{\twopartdef}[4]
{
	\left\{
		\begin{array}{ll}
			#1 & \mbox{,  if } #2 \\
			#3 & \mbox{, if } #4
		\end{array}
	\right.
}


\title{Introdu\c c\~ao \`a An\'alise no $\mathbb R^n$ \\ Prova $1$}
%Partially hyperbolic diffeomorphims with Null Central Lyapunov exponents and non-compact %central leaves}
\author{Prof.Gabriel Ponce \\ {\small IMECC- UNICAMP} }

\date{}      
                                    
\begin{document}

\begin{center}
{\Large Matem\'atica IV  - Fun\c c\~oes de Vari\'aveis Complexas} \\
\vspace{.2cm}
{Prof. Gabriel Ponce}
\end{center}

\noindent  \textbf{RA (Leg\'ivel)} : 
\quad 
\vspace{.5cm}

\begin{center}
\vspace{0.5cm}
\begin{tabular}{r|r|r|r|r|r|r|r|r|r}
 
1 & 2 & 3 & 4 & 5 & Total \\ % Note a separação de col. e a quebra de linhas
\hline                               % para uma linha horizontal
 \quad  \quad &\quad  \quad &\quad  \quad&\quad  \quad&\quad  \quad &\quad  \quad 
\end{tabular}
\end{center}

\noindent {\bf Observa\c c\~ao:} 
\begin{itemize}
\item[1)] Este simulado \'e formado atrav\'es de um banco de problemas selecionado pelo docente e n\~ao necessariamente tem rela\c c\~ao com formato, n\'ivel de dificuldade, ou qualquer outro aspecto das provas de MA044/Fun\c c\~oes de Vari\'avel Complexa na sua disciplina. Use este simulado apenas como uma forma adicional de estudo.
\item[2)] N\~ao se esque\c ca de verificar as hip\'oteses dos teoremas necess\'arios antes de aplica-los.
\item[3)] Justifique bem suas solu\c c\~oes.
\end{itemize}



\newpage

\begin{center}
{\Large Simulado de Matem\'atica IV  - Fun\c c\~oes de Vari\'aveis Complexas} \\
\vspace{.2cm}
{Prof. Gabriel Ponce}
\end{center}

\vspace{1cm}

 \begin{enumerate}
\item \begin{description} 
\item[a)] (1.5) Seja $z=1+\sqrt{3}\cdot i$. 
Determine  $z^{2019}$. \\
{\small Obs: coloque o resultado final na forma $x+iy$.}


\item[b)] (1.0) Sejam $a,b,c \in \mathbb C$ tais que:
\[|c-a| = |c-b| = \frac{|a-b|}{2}.\]
Prove que, para todo n\'umero complexo $z\in \mathbb C$ temos:
\[|z-a| \cdot |z-b| \geq \left(|z-c| - \frac{|a-b|}{2} \right)^2\]

\end{description}



\item Prove que dois n\'umeros complexos n\~ao nulos $z_1$ e $z_2$ tem o mesmo m\'odulo se, e somente se, existem n\'umeros complexos $c_1$ e $c_2$ tais que $z_1 = c_1c_2$ e $z_2=c_1\overline{c_2}$.


\item \begin{itemize}
\item[a)] Use a f\'ormula binomial e a f\'ormula de Moivre para escrever
\[\cos n\theta +i\operatorname{sen} n\theta = \sum_{k=0}^{n}\binom{n}{k}\cos^{n-k}\theta(i \operatorname{sen}\theta)^k, \]
$n=0,1,2,...$.
Defina o inteiro $m$ da seguinte forma: $m=n/2$ se $n$ \'e par e $m=(n-1)/2$ caso $n$ seja \'impar. Mostre que
\[\cos n\theta = \sum_{k=0}^{m}\binom{n}{2k}\cos^{n-2k}\theta \cdot (-1)^k\cdot \operatorname{sen}^{2k}\theta,\]
$n=0,1,2,...$.

\item[b)] Escreva $x= \cos \theta$ na somat\'oria do final da parte $(a)$. Mostre que fazendo essa substitui\c c\~ao, $\cos n \theta$ vira um polin\^omio em $x$:
\[T_n(x) = \sum_{k=0}^{m}\binom{n}{2k}x^{n-2k}(1-x^2)^{k},\]
$n=0,1,2,...$. 
\end{itemize}




\end{enumerate}

 \end{document}