
\problem
Defina o conjunto dos n\'umeros complexos e a opera\c c\~ao de multiplica\c c\~ao de n\'umeros complexos.



\problem
 Verifique que 
\begin{itemize}
\item[a)] $(\sqrt{2} - i) - i(1-\sqrt{2}i) = -2i$
\item[b)] $(2,-3)(-2,1)=(-1,8)$.
\end{itemize}



\problem
Mostre que $\operatorname{Re}(iz) = -\operatorname{Im}(z)$ e que $ \operatorname{Im}(iz) = \operatorname{Re}(z)$.



\problem
Mostre que a opera\c c\~ao de multiplica\c c\~ao de n\'umeros complexos \'e comutativa.


\problem
Use a lei associativa da adi\c c\~ao e a lei distributiva para mostrar que
\[z(z_1+z_2+z_3) = zz_1+zz_2+zz_3.\]



\problem
Reduza as seguintes quantidades:
\begin{description}
\item[i)] $\frac{1+5i}{3-4i}$
\item[ii)] $\frac{3+2i}{1+i} + \frac{7-i}{1-i}$
\item[iii)]$\frac{2i}{2+i}\cdot \frac{2-i}{1-i}$.
\end{description}



\problem
Mostre que se $z_1z_2z_3=0$ ent\~ao pelo menos um dos n\'umeros complexos $z_1,z_2,z_3$ \'e nulo.

\problem
Para cada um dos casos, localize os n\'umeros $z_1+z_2$ e $z_1-z_2$ vetorialmente:
\begin{itemize}
\item[i)] $z_1 = 2i$, $z_2 = \frac{2}{3} - i$;
\item[ii)] $z_1 = i$, $z_2= 3$;
\item[iii)] $z_1 = x+iy$, $z_2 = x-iy$.
\end{itemize}

\problem
Utilize as propriedades de m\'odulo j\'a demonstradas em sala para provar que, dados quaisquer n\'umeros complexos $z_1,z_2,z_3,z_4$ com $|z_3| \ne |z_4|$ tem-se:
\[\frac{\operatorname{Re}(z_1+z_2)}{|z_3+z_4|} \leq \frac{|z_1|+|z_2|}{||z_3|-|z_4||}.\]

\problem
Mostre que para todo $z\in \mathbb C$ temos $\sqrt{2} |z| \geq |\operatorname{Re}(z)| + |\operatorname{Im}(z)|$.

\problem
Desenhe o conjunto dos pontos dado pela condi\c c\~ao:
\begin{itemize}
\item[a)] $|z-1+i| =1$
\item[b)] $|z+2i| \leq 4$
\item[c)] $|2z-2+i| \geq 7$.
\end{itemize}

\problem
Usando as propriedades de conjugados e m\'odulos estabelecidas em sala mostre que
\begin{itemize}
\item[a)] $\overline{\overline{z} +3i} = z-3i$
\item[b)] $|(2\overline{z} +5)(\sqrt{2}-i) = \sqrt{3} | 2z+5|$.
\end{itemize}

\problem
Desenhe o conjunto dos pontos determinado pela condi\c c\~ao:
\begin{itemize}
\item[a)] $\operatorname{Re}(\overline{z}-i) = 2$;
\item[b)] $|2\overline{z} +i| = 4$.
\end{itemize}

\problem
Mostre que quando $z_2$ e $z_3$ s\~ao n\~ao nulos temos
\[\overline{ \left( \frac{z_1}{z_2z_3} \right) } = \frac{\overline{z_1}}{\overline{z_2}\overline{z_3}}.\]

\problem
Mostre que se $|z|=2$ ent\~ao
\[\left|  \frac{1}{z^4-4z^2+3}  \right| \leq \frac{1}{3}.\]


\problem
Mostre que a hip\'erbole $x^2 - y^2 = 1$ pode ser escrita na forma
\[z^2+\bar{z}^2 =2.\]

\problem
Encontre o argumento principal $\operatorname{Arg} z$ quando
\begin{itemize}
\item[a)] $z= \frac{i}{-2-2i}$
\item[b)] $z= (\sqrt{3}-i)^6$
\end{itemize}


\problem
Mostre que $|e^{i\theta}| =1$ e que $\overline{e^{i\theta}} = e^{-i\theta}$.

\problem
Usando a forma exponencial mostre que
\begin{itemize}
\item[a)] $i(1-\sqrt{3}i)(\sqrt{3}+i)=2(1+\sqrt{3}i)$
\item[b)] $(-1+i)^7 = -8(1+i)$
\item[c)]$(1+\sqrt{3}i)^{-10} = 2^{-11}(-1+\sqrt{3}i)$.
\end{itemize}


\problem
Mostre que se $\operatorname{Re}(z_1) >0 $ e $\operatorname{Re}(z_2) >0$ ent\~ao
\[\operatorname{Arg}(z_1z_2) = \operatorname{Arg}(z_1)+\operatorname{Arg}(z_2),\]
onde os argumentos principais s\~ao usados.

\problem
Prove que dois n\'umeros complexos n\~ao nulos $z_1$ e $z_2$ tem o mesmo m\'odulo se, e somente se, existem n\'umeros complexos $c_1$ e $c_2$ tais que $z_1 = c_1c_2$ e $z_2=c_1\overline{c_2}$.

\problem
Mostre que para todo $z \ne 1$ temos
\[1+z+z^2+...+z^n = \frac{1-z^{n+1}}{1-z}.\]

\problem
Utilize o resultado do problema anterior para provar que dado qualquer $0<\theta < 2\pi$ temos
\[1+\cos \theta + \cos 2\theta + ... + \cos n\theta = \frac{1}{2}+\frac{\operatorname{sen}[(2n+1)\theta/2]}{2\operatorname{sen}(\theta/2)}.\]

\problem 
\begin{itemize}
\item[a)] Use a f\'ormula binomial e a f\'ormula de Moivre para escrever
\[\cos n\theta +i\operatorname{sen} n\theta = \sum_{k=0}^{n}\binom{n}{k}\cos^{n-k}\theta(i \operatorname{sen}\theta)^k, \]
$n=0,1,2,...$.
Defina o inteiro $m$ da seguinte forma: $m=n/2$ se $n$ \'e par e $m=(n-1)/2$ caso $n$ seja \'impar. Mostre que
\[\cos n\theta = \sum_{k=0}^{m}\binom{n}{2k}\cos^{n-2k}\theta \cdot (-1)^k\cdot \operatorname{sen}^{2k}\theta,\]
$n=0,1,2,...$.

\item[b)] Escreva $x= \cos \theta$ na somat\'oria do final da parte $(a)$. Mostre que fazendo essa substitui\c c\~ao, $\cos n \theta$ vira um polin\^omio em $x$:
\[T_n(x) = \sum_{k=0}^{m}\binom{n}{2k}x^{n-2k}(1-x^2)^{k},\]
$n=0,1,2,...$. 
\end{itemize}



\problem
Encontre as ra\'izes quadradas e expresse em coordenadas retangulares
\begin{itemize}
\item[a)] $2i$
\item[b)] $1-\sqrt{3}i$.
\end{itemize}  


\problem
 Em cada caso, encontre todas as ra\'izes em coordenadas retangulares e exiba elas como v\'ertices de um certo pol\'igono regular. Al\'em disso, identifique a raiz principal:
\begin{itemize}
\item[a)]$(-1)^{1/3}$
\item[b)] $8^{1/6}$.
\end{itemize}  


\problem
 Encontre os quatro zeros do polin\^omio $z^4+4$, um deles sendo $z_0 = \sqrt{2}e^{i\pi/4}=1+i$. Ent\~ao, use estes quatro zeros para fatorar $z^4+4$ em fatores quadr\'aticos com coeficientes reais.


\problem
 Mostre que se $c$ \'e qualquer raiz $n$-\'esima  da unidade, ent\~ao
\[1+c+c^2 +...+c^{n-1} = 0.\]



\problem
 Mostre que a f\'ormula de Bhaskara usual resolve equa\c c\~oes quadr\'aticas complexas:
\[az^2+bz+c=0\]
onde $a\ne0$ e $a,b,c\in \mathbb C$. Ou seja, mostre que as solu\c c\~oes s\~ao dadas por 
\[z=\frac{-b+(b^2-4ac)^{1/2}}{2a}\]
onde ambas as ra\'izes quadradas devem ser consideradas quando $b^2-4ac \ne 0$. Use este resultado para encontrar as ra\'izes da equa\c c\~ao: $z^2+2z+(1-i) = 0$.

\problem 
\begin{description}
\item[a)] (1.0) Seja $z= 1+i$, $w = \frac{-\sqrt{3}}{2}+\frac{1}{2}i$, calcule $z^{12} \cdot w^{-8}$ (Obs: coloque na forma $x+iy$).

\item[b)] (1.0) Mostre que se $z\in \mathbb C$ \'e tal que $|z|=2$, ent\~ao
\[\left|  \frac{1}{z^4-4z^2+3}  \right| \leq \frac{1}{3}.\]
\end{description}


\problem
\begin{itemize}
\item[a)] (1.0) Seja $z= 1+i$, $w = \frac{-\sqrt{3}}{2}+\frac{1}{2}i$, calcule $z^{12}\cdot w^{-8}$ (Obs: coloque na forma $x+iy$).

\item[b)] (1.0) Mostre que se $c$ \'e qualquer raiz $n$-\'esima  da unidade, ent\~ao ou $c=1$ ou
\[1+c+c^2 +...+c^{n-1} = 0.\]
\end{itemize}

\problem
Considere um ponto $z\in \mathbb C$ qualquer. Lembre-se que $|z-a|$ \'e a dist\^ancia de $z$ \`a $a$ e $|z-b|$ \'e a dist\^ancia de $z$ \`a $b$. 

\problem
\begin{description}
\item[a)] (1.5) Seja $w= \frac{1+i}{1-i}$ tome $z=\frac{\sqrt{2}}{2}w + \frac{\sqrt{2}}{2}\cdot i \cdot w$.
Calcule a \textbf{ra\'iz c\'ubica principal} de $z$. \\
{\small Obs: coloque o resultado final na forma $x+iy$}

\item[b)] (1.0) Seja $C_0$ o c\'irculo no plano complexo com centro em $z_0=i$ e raio $R=1$, mostre que 
\[\left| z^2-z(2i+1)+i  \right| \leq 3, \quad \forall \text{ } z\in C_0\]
\end{description}



\problem
\begin{description} 
\item[a)] (1.5) Seja $w= \frac{1+i}{1-i}$ tome $z=\frac{1}{2}\cdot (1+w)^2$. 
Determine o conjunto $z^{1/3}$ e indique qual \'e a \textbf{ra\'iz c\'ubica principal} de $z$. \\
{\small Obs: coloque os resultados finais na forma $x+iy$.}

\item[b)] (1.0) Seja $C_0$ o c\'irculo no plano complexo com centro em $z_0=-2$ e raio $R=2\sqrt{5}$, mostre que 
\[\left| z^2+z(2-i)-2i  \right| \geq 10, \quad \forall \text{ } z\in C_0.\]
\end{description}


\problem
\begin{itemize}
\item[a)] (1.5) Seja $z=\frac{1}{2}+\frac{\sqrt{3}}{2}\cdot i$. 
Determine  $z^{30}$. \\
{\small Obs: coloque o resultado final na forma $x+iy$.}

\item[b)] (1.0) Sejam $z_1,z_2 \in \mathbb C \setminus \{0\}$. Prove que $|z_1|=|z_2|$ se, e somente se, existem n\'umeros complexos $c_1$ e $c_2$ tais que
\[z_1=c_1\cdot c_2 \quad \text{e} \quad z_2 = c_1\cdot \overline{c_2}.\]
\end{itemize}


\problem
\begin{description} 
\item[a)] (1.5) Seja $z=1+\sqrt{3}\cdot i$. 
Determine  $z^{2019}$. \\
{\small Obs: coloque o resultado final na forma $x+iy$.}


\item[b)] (1.0) Sejam $a,b,c \in \mathbb C$ tais que:
\[|c-a| = |c-b| = \frac{|a-b|}{2}.\]
Prove que, para todo n\'umero complexo $z\in \mathbb C$ temos:
\[|z-a| \cdot |z-b| \geq \left(|z-c| - \frac{|a-b|}{2} \right)^2\]

\end{description}


\problem
\begin{description}
%\item[a)] (1.0) Seja $w=\frac{\sqrt{3}+1+i(1-\sqrt{3})}{1-i}$, determine a forma polar de $w$ e seu argumento principal.
\item[a)] (1.5) Considerando $z=-1+i$ determine $z^{1000}$ e a ra\'iz c\'ubica \textbf{principal} de $z$.
\item[b)] (1.0) Seja $C_0$ o c\'irculo no plano complexo com centro em $z_0=i$ e raio $R=2$, mostre que 
\[\left| 2z^2-4zi-1  \right| \geq 7, \quad \forall \text{ } z\in C_0\]
\end{description}



